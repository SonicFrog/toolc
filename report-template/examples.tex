Give code examples where your extension is useful, and describe how they work
with it. Make sure you include examples where the most intricate features of
your extension are used, so that we have an immediate understanding of what the
challenges are.

You can pretty-print tool code like this:
\begin{lstlisting}
object {
  def main() : Unit = { println(new A().foo(-41)); }
}

class A {
  def foo(i : Int) : Int = {
    var j : Int;
    if(i < 0) { j = 0 - i; } else { j = i; }
    return j + 1;
  }
}
\end{lstlisting}

This section should convince us that you understand how your extension can be
useful and that you thought about the corner cases.

\subsection{IO Functions}
Tool's greatest limitation is its lack of interaction with the user. JavaScript, on the  contrary, has some already implemented functions for all those purposes.
\subsubsection{Input gestion}
\subsubsection{Output gestion}

\subsection{New Types}
Another great limitation of Tool is that it only support very basic types. Which is why the support for Double and any type of Array has been added to the grammar of Tool.
\subsubsection{Double}
Integer are now defined as a subtype of double for simplification purposes.
You can concatenate a Double with a String.
\subsubsection{Array of any Types}
\subsubsection{Array of more than one dimension}
If you want to see a real world use of a two dimension matrix, you can check the file \emph{programs/Sudokupdated.tool} which is an updated version of the given \emph{Sudoku.tool} implementing the support
\subsection{Parametrized constructor}


