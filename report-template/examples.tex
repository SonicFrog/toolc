\subsection{IO Functions}
Tool's greatest limitation is its lack of interaction with the
user. JavaScript, on the  contrary, has some already implemented
functions for all those purposes. All these IO functions are managed
through the use of new built-in IO object in order to reduce the
number of keywords required.

\subsubsection{Input management}

\begin{table}[h!]
  \centering
  \begin{tabular}{ll}
    \toprule
    Name & Role \\
    \midrule
    readDouble & Allows the user to enter a Double \\
    readString & Allows the user to enter String \\
    readInteger & Allows the user to enter an Integer \\
  \end{tabular}
  \caption{IO methods}
  \label{tab:infuncs}
\end{table}

All these methods are meant to be used as expressions and can be
combined as any expressions as shown below.

\begin{lstlisting}
  var b : Int;
  var c : Double;

  b = IO.readInteger();
  c = b + IO.readDouble();
\end{lstlisting}

\subsubsection{Output management}

Running inside of a browser allows for two kinds of output. We can
either show a popup or write on the webpage. Both these methods are
supported as shown in~\autoref{tab:outfuncs}. For debugging purposes,
it is also possible to log message to the Javascript console.

\begin{table}[h!]
  \centering
  \begin{tabular}{ll}
    \toprule
    Name & Role \\
    \midrule
    log & Logs a message in the browser's console \\
    writeLine & Writes a line on the webpage \\
    showPopup & Creates a popup in the browser \\
  \end{tabular}
  \caption{Output methods}
  \label{tab:outfuncs}
\end{table}

\begin{lstlisting}
  println(``Hello''); //Compatibility
  IO.showPopup(10 + ``Some string'');
  IO.writeLine(``<strong>Some HTML!</strong>'');
\end{lstlisting}

\subsection{New Types}

Another great limitation of Tool is that it only supports very basic
types. Our extension adds support for generic arrays and floating
point numbers to Tool.

\begin{lstlisting}
  var c : Double;
  c = 11.2;
  IO.showPopup(``Here's a Double: '' + c);
\end{lstlisting}


\subsubsection{Double}

In order to facilitate manipulations of basic types Integer are
considered to be a subtype of Double. Indeed you can view an Integer
as a Double with a zero decimal part. It is thus possible to add an
Integer with a Double and the result will be a Double.

\begin{lstlisting}
  var c : Integer;
  var d : Double;

  c = 10;
  d = 11.2;

  println(``This is ''c + d); // This is 21.2
\end{lstlisting}

\subsubsection{Array of any Types}

Generic arrays are also part of the extension. It is now possible to
instantiate arrays of any types, including Object types.

\begin{lstlisting}
  class Hello {
    def method() : Int = {
      return 10;
    }
  }

  def someMethod() : Integer = {
    var array : Hello[];
    array = new Hello[5];
    array[2] = new Hello();
    return array[2].method();
  }
\end{lstlisting}

\subsubsection{Array of more than one dimension}
If you want to see a real world use of a two dimension matrix, you can
check the file \emph{programs/Sudokupdated.tool} which is an updated
version of the given \emph{Sudoku.tool} making use of the
multi-dimensional arrays.

\begin{lstlisting}
  var d : Double[][]; //Matrix of Double
  var i : Integer;

  d = new Double[3];
  i = 0;
  while(i < 3) {
    d[i] = new Double[3];
  }
  println(d[0][2]); //d is a 3x3 matrix of Doubles
\end{lstlisting}

\subsection{Parametrized constructor}

The support for constructors has also been added to the Tool
language. However currently only one constructor per class is
allowed. The constructor must be declared directly after the fields of
the class and before any method declaration. In order to retain
compatibility with older Tool programs a class declaring no
constructor is assumed to have a default constructor which takes no
arguments.

The following code shows how to declare a constructor:
\begin{lstlisting}
  class A {
    var field1 : Int;
    var field2 : Double;

    this(param1 : Int, param2 : Double) = {
      field1 = param1;
      field2 = param2;
    }
  }

  var a : A;
  a = new A(10, 11.2);

  class B {
    //B automatically has a default constructor
  }

  var b : B;
  b = new B(); //Using the default constructor
\end{lstlisting}
