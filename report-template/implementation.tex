This section will summarize all the implementation details we had to implement after a quick
look to the required theoretical background.

\subsection{Theoretical Background}
While not using specific theoretical concepts none of us had ever done any programming in JavaScript.
The first task was then to learn about the specifications of this language in order to be able to identify the ``translation" we'd have to make. \cite{JSGood} gave us a general picture about the ins-and-outs of JavaScript.
And \cite{JSProto} was filled with details about the prototypes in JavaScript which we had to use as there is no classes in that delightful language.

\subsection{Implementations Details}
\subsubsection{Tool Classes and inheritance}
As already mentioned, JavaScript does not have Classes as of, instead it has an element called Object which you can define a prototype of and then define a function associated to a prototype. Heritage its self can be simulated with the method \emph{Object.create(prototype)} which will take the \emph{prototype} as a base for our child class.The following code snippet represent a simple heritage in Tool:
\begin{lstlisting}
class Slot {
	var value: Int;

	this(val : Int) = {
		value = val;
	}

	def getVal() : Int = {
		return value;
	}

	def isInit() : Bool = {
		return false;
	}

	def setVal(val : Int) : Slot = {
		value = val;
		return this;
	}
}

class InitSlot extends Slot{

	this(val : Int) = {
		value = val;
	}

	def isInit() : Bool = {
		return true;
	}
}
\end{lstlisting}

And the following snippet is the Javascript code generated from the previous Tool code:
\begin{lstlisting}[language=javascript]
function Slot(val){
    this.value = val;
}
Slot.prototype.getVal = function() {
    return this.value;
}
Slot.prototype.isInit = function() {
    return false;
}
Slot.prototype.setVal = function(val) {
    this.value = val;
    return this;
}

InitSlot.prototype = Object.create(Slot.prototype);
function InitSlot(val){
    this.value = val;
}
InitSlot.prototype.isInit = function() {
    return true;
}

\end{lstlisting}

\subsubsection{Imitating the environnement of a shell}
A lot of programs in the set you've given us are pretty printing things in the console
(e.g. Sudoku.tool) we made sure the formatting in the web page would be like a shell window,
using a monospace font (Menlo) and having the whole output of the script inside <pre></pre> html tags to ensure that the formatting and other special chars are displayed correctly.
\subsubsection{Input management}
JavaScript has a built in \emph{prompt()} method but this does carry Type information so as our Tool is strongly typed, we have created three input reading methods (as you've seen in Table 1.) which are all translated to a \emph{prompt()} call but we use the functions \emph{parseInt()} or \emph{parseFloat()} on the result of the method call to ensure we have either integer or double and thus respect tool's strong typing.


\subsubsection{Double in Tool and Integer in JavaScript}
Integer are now defined as a subtype of double so it is possible to do the arithmetic operations on an integer and a double without any trouble thanks to the weak typing of JavaScript.

But this introduce a new problem, as there is only a number type (IEEE 754 64-bits standard) there is no integer in absolute. This is problematic, for example when you want to do an integer by integer division
in JavaScript the result will not be an integer so we had to make a special case for that kind of division and call \emph{Math.floor()} over the division.

But there is still a major drawback which is that int/int division have much less bits to work with than
in the jvm so you get wrong results faster when having huge integer with a decimal part (e.g. \emph{Maze.tool} cannot be ran because its operations are out of the range of JavaScript capabilites, we tried to fix the program itself but whoever coded it created a monster! (e.g. (a * 65536) / 65536 ... what's the point?)  

\subsubsection{Array of any Types (including arrays of arrays)}
To be able to handle multiple types of array we had to build a similar architecture for the arrays than the one existing for the TObject. We then have a type named TGenericArray which ensure that we have indeed an array and the type TArray which has a field innerType describing the nature of the array.
This way, we are able to have arrays of any types and the sub-typing is done by comparing the types of the innerTypes.
Lexing and parsing had to be changed in order to support any number of brackets (i.e. \emph{superArray[][][][][]}) and to support method call on array reading (i.e. \emph{superArray[2].myMethod()}).

\subsubsection{Parametrized constructor}
The syntax for constructor declaration was designed with the primary
goal of keeping the Tool grammar LL(1). Initially, the support for
multiple constructors was planned but Javascript's handling of
constructor does not allow for constructor overloading.

\subsubsection{Legacy support}
As the number of developper using our modified syntax of Tool is really small, we were obligated to have legacy support for the original Tool syntax in order to have more that five programs to test!
That's why we kept the lexing of println() but we read it as a token WriteLine to avoid code duplication.
