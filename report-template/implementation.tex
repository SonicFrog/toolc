This section will summarize all the implementation details we had to implement after a quick
look to the required theoretical background.

\subsection{Theoretical Background}
While not using specific theoretical concepts none of us had ever done any programming in JavaScript.
The first task was then to learn about the specifications of this language in order to be able to identify the ``translation" we'd have to make. \cite{JSGood} gave us a general picture about the ins-and-outs of JavaScript.
And \cite{JSProto} was filled with details about the prototypes in JavaScript which we had to use as there is no classes in that delightful language.

\subsection{Implementations Details}
\subsubsection{Tool Classes and inheritance}
\subsubsection{Imitating the environnement of a shell}
\subsubsection{Input gestion}
\subsubsection{Double in Tool and Integer in JavaScript}
Integer are now defined as a subtype of double for simplification purposes.
You can concatenate a Double with a String.
\subsubsection{Array of any Types (including arrays of arrays)}
\subsubsection{Parametrized constructor}
\subsubsection{Legacy support}



Describe all non-obvious tricks you used. Tell us what you thought was hard and
why. If it took you time to figure out the solution to a problem, it probably
means it wasn't easy and you should definitely describe the solution in details
here. If you used what you think is a cool algorithm for some problem, tell us.
Do not however spend time describing trivial things (we what a tree traversal
is, for instance). 

After reading this section, we should be convinced that you knew what you were
doing when you wrote your extension, and that you put some extra consideration
for the harder parts.

